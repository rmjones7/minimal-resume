%%%%%%%%%%%%%%%%%%%%%%%%%%%%%%%%%%%%%%%%%%%%%%%%%%%%%%%%%%%%%%%%%%%%%%%%
%%%%%%%%%%%%%%%%%%%%%% Simple LaTeX CV Template %%%%%%%%%%%%%%%%%%%%%%%%
% _author_: Mitchell Jones
% _email_: rmjones7@illinois.edu
% forked from minimal-resume (github.com/arunlakshmanan)
% modified from the Simple LaTeX CV Template (github.com/kscottz)
%%%%%%%%%%%%%%%%%%%%%%%%%%%%%%%%%%%%%%%%%%%%%%%%%%%%%%%%%%%%%%%%%%%%%%%%

%%%%%%%%%%%%%%%%%%%%%%%%%%%% Document Setup %%%%%%%%%%%%%%%%%%%%%%%%%%%%

\documentclass[10pt]{article}

% Input details here:
\newcommand{\authorname}{Robert Mitchell Jones}
\newcommand{\tel}{(860) 670-4546}
\newcommand{\addr}{Urbana IL, 61801}
\newcommand{\email}{\href{mailto:rmjones7@illinois.edu}{rmjones7@illinois.edu}}
\newcommand{\github}{\href{http://github.com/rmjones7}{github.com/rmjones7}}
\newcommand{\website}{\href{www.robertmitchelljones.com}{robertmitchelljones.com}}

\usepackage{natbib}
\usepackage{bibentry}

% This is a helpful package that puts math inside length specifications
\usepackage{calc}
\usepackage{textcomp}

% Layout: Puts the section titles on left side of page
\reversemarginpar

% Setup colors
\usepackage{color}
\definecolor{darkblue}{rgb}{0.0,0.0,0.3}
\definecolor{lightgray}{rgb}{0.5,0.5,0.5}

%% Use these lines for letter-sized paper
\usepackage[paper=letterpaper,
            %includefoot, % Uncomment to put page number above margin
            marginparwidth=1.00in,     % Length of section titles
            marginparsep=.05in,       % Space between titles and text
            margin=0.75in,
            includemp]{geometry}

%% More layout: Get rid of indenting throughout entire document
\setlength{\parindent}{0in}

%% This gives us fun enumeration environments. compactitem will be nice.
\usepackage{paralist}

%% Last updated footer
\usepackage{fancyhdr}
\usepackage[en-US]{datetime2}

\fancyhf{}
\renewcommand{\headrulewidth}{0pt}
\fancyfoot[C]{\color{lightgray}\footnotesize Last updated on \today}
\pagestyle{fancy}
\pagenumbering{gobble}

% Finally, give us PDF bookmarks
\usepackage{hyperref}
\hypersetup{colorlinks,breaklinks,pageanchor=false,
            linkcolor=darkblue,urlcolor=darkblue,
            anchorcolor=darkblue,citecolor=darkblue}

%%%%%%%%%%%%%%%%%%%%%%%% End Document Setup %%%%%%%%%%%%%%%%%%%%%%%%%%%%


%%%%%%%%%%%%%%%%%%%%%%%%%%% Helper Commands %%%%%%%%%%%%%%%%%%%%%%%%%%%%

% The title (name) with a horizontal rule under it
%
% Usage: \makeheading{name}
%
% Place at top of document. It should be the first thing.

\newcommand{\makeheading}%
         {\vspace*{-2.5em}%
          \hspace*{-\marginparsep minus \marginparwidth}%
             \begin{minipage}[t]{\textwidth+\marginparwidth+\marginparsep}%
               {\Huge \hspace{-1.20em} \authorname}%
             \end{minipage}
          \hfill%
             \begin{minipage}[t]{\textwidth}%
               \begin{flushright}%
                 \vspace{-4.0em}%
                 \tel \\
                 \email \\
                 \github \\
                 \website
               \end{flushright}
             \end{minipage}}

% The section headings
%
% Usage: \section{section name}
%
% Follow this section IMMEDIATELY with the first line of the section
% text. Do not put whitespace in between. That is, do this:
%
%       \section{My Information}
%       Here is my information.
%
% and NOT this:
%
%       \section{My Information}
%
%       Here is my information.
%
% Otherwise the top of the section header will not line up with the top
% of the section. Of course, using a single comment character (%) on
% empty lines allows for the function of the first example with the
% readability of the second example.
\renewcommand{\section}[2]%
        {\pagebreak[2]\vspace{1.3\baselineskip}%
         \phantomsection\addcontentsline{toc}{section}{#1}%
         \hspace{0in}%
         \marginpar{
         \raggedright \scshape #1}#2}

% An itemize-style list with lots of space between items
\newenvironment{outerlist}[1][\enskip\textbullet]%
        {\begin{itemize}[#1]}{\end{itemize}%
         \vspace{-.6\baselineskip}}

% An environment IDENTICAL to outerlist that has better pre-list spacing
% when used as the first thing in a \section
\newenvironment{lonelist}[1][\enskip\textbullet]%
        {\vspace{-\baselineskip}\begin{list}{#1}{%
        \setlength{\partopsep}{0pt}%
        \setlength{\topsep}{0pt}}}
        {\end{list}\vspace{-.6\baselineskip}}

% An itemize-style list with little space between items
\newenvironment{innerlist}[1][\enskip\textbullet]%
        {\begin{compactitem}[#1]}{\end{compactitem}}

% To add some paragraph space between lines.
% This also tells LaTeX to preferably break a page on one of these gaps
% if there is a needed pagebreak nearby.
\newcommand{\blankline}{\quad\pagebreak[2]}

% Software macros
\newcommand \CPP        {{C\nolinebreak[4]\hspace{-.05em}\raisebox{.4ex}{\tiny\bf ++}}}

%%%%%%%%%%%%%%%%%%%%%%%% End Helper Commands %%%%%%%%%%%%%%%%%%%%%%%%%%%

%%%%%%%%%%%%%%%%%%%%%%%%% Begin CV Document %%%%%%%%%%%%%%%%%%%%%%%%%%%%

\begin{document}
\nobibliography{ref}
\bibliographystyle{abbrvnat}

\makeheading

\section{Objectives}
I want to work with a community of researchers to put to use my skills in robotics and control systems and build creative solutions in the world of autonomous systems.

\section{Education}
\textbf{MS Mechanical Science and Engineering}   \hfill \textbf{Sept 2015 - }\textbf{present}\\
\textit{\href{http://www.mechse.illinois.edu/}
        {University of Illinois at Urbana-Champaign},
        Urbana, IL.} \bigskip

\textbf{BS Mechanical Engineering}  \hfill \textbf{Sept 2011 - May 2015}\\
\textit{\href{http://uri.edu/}
        {University of Rhode Island},
        Kingston, RI.}
\blankline
%%%%%%%%%%%%%%%%%%%%%%%%%%%%%%%%% Page break

\section{Relevant Coursework}
\textit{Robust Adaptive Control, Optimum Control Systems, Advanced Robotics Planning, Analysis of Nonlinear Systems, Control of Complex Systems, Control Systems Theory \& Design.}
\blankline

%%%%%%%%%%%%%%%%%%%%%%%%%%%%%%%%% Page break

\section{Experience}
\href{http://naira.mechse.illinois.edu/}{\textbf{Advanced Controls Research Laboratory}, Urbana, IL.}
\hfill \textbf{Aug 2014 - }\textbf{present} \\
\textit{Graduate Research Assistant}%
\begin{outerlist}
\item Designed a computationally efficient trajectory generation approach using piecewise B\'ezier curves for differentially flat systems. This approach can be used to generate feasible minimum snap trajectories for quadrotors in the least time with the added advantage of using the convex hulls of B\'ezier curves to check for any collisions incurred during interpolation. Further analysis and results can be found in the master's thesis.
\item Constantly involved with all software-related development in the research group. Implemented path following controllers on ground robots for precise tracking, designed line-of-sight based collision avoidance methods replying purely on directional sensor information, and more recently, geometric controllers were implemented to track aggressive trajectories on quadrotors.
\end{outerlist}
\blankline

\href{https://ti.arc.nasa.gov/}{\textbf{NASA Ames Research Center}, Moffett Field, CA.}
\hfill \textbf{June 2016 - Aug 2016} \\
\textit{Graduate Intern}%
\begin{outerlist}
\item Designed and built an aerial manipulator system for use on a small quadrotor.
\item Programmed an onboard controller for execution of manipulator commands during flight.
\item Modeled manipulator dynamics in Simulink and developed method for
stabilizing vehicle during normal operation and highly dynamic maneuvers.
\end{outerlist}
\blankline

\href{http://greensightag.com/}{\textbf{Greensight Agronomics}, Boston, MA.}
\hfill \textbf{June 2015 - Sept 2015} \\
\textit{Mechanical Engineer \& UAV Pilot}%
\begin{outerlist}
\item Developed subsystems for an early-stage startup company delivering
autonomous plant monitoring solutions to the agricultural and recreational industries.
\item Flew contracted on-site missions mapping various farms, golf courses, athletic fields, etc.
\end{outerlist}
\blankline

%%%%%%%%%%%%%%%%%%%%%%%%%%%%%%%%% Page break

\section{Technical Summary}
I have extensive experience with robotics development on quadrotors and ground robots in the realms of controller architecture design, motion planning and trajectory generation. I am very interested in computationally efficient solutions to trajectory generation for mobile robots in cluttered environments. \medskip

I am adept at writing \textbf{C}/\textbf{\CPP} software for embedded systems, \textbf{Python} for rapid prototyping and high-level decision making, \textbf{ROS} packages for communication services, and \textbf{MATLAB} and \textbf{Simulink} models for analysis and design. I have primarily developed software for ARM-based processors running Linux/FreeRTOS environments.

\newpage
\section{Publications}
\begin{lonelist}
\item \bibentry{Jones_undated-hi}
\item \bibentry{Marinho2016-fo}
\end{lonelist}

\end{document}


%%%%%%%%%%%%%%%%%%%%%%%%%% End CV Document %%%%%%%%%%%%%%%%%%%%%%%%%%%%%
